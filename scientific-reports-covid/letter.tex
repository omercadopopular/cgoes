\documentclass{article}
\usepackage[utf8]{inputenc}
\usepackage{mathtools}
\usepackage{amsmath}
\usepackage{amsthm}
\usepackage{amsfonts}
\usepackage{amssymb}
\usepackage{comment}
\usepackage{appendix}
\usepackage{units}
\usepackage{minted}
\usepackage[affil-it]{authblk}

\usepackage[style=nature]{biblatex}
\addbibresource{ref.bib}

\usepackage{geometry}
\usepackage{csquotes}

\theoremstyle{lemma}
\newtheorem*{lemma}{Lemma}

\newtheorem{assumption}{Assumption}
\newtheorem{proposition}{Proposition}
%\newtheorem{lemma}{Lemma}
\newtheorem{definition}{Definition}

\DeclareMathOperator*{\argmax}{arg\,max}
\DeclareMathOperator*{\argmin}{arg\,min}


\title{Explanation of changes regarding ``Pairwise difference regressions are just weighted averages''}
\author{Carlos Góes\footnote{University of California -- San Diego, Department of Economics. cgoes@ucsd.edu}}
\date{\today}

\begin{document}

\maketitle

Here I summarize the main changes done to the Matters Arising commentary. They are all stylistic or small clarifications that do not change anything substantial in the paper.

\begin{itemize}
    \item \textbf{Change in title}. I changed the title from ``Savaris et al (2021) erroneously interpreted their regressions'' to ``Pairwise difference regressions are just weighted averages,'' as recommended by the Editors.
    \item \textbf{Nonstationarity of variables}. In their response, the authors took issue with the simplifying assumption that $X^i_t, Y_t^i$ being iid. The derivations of the paper, however, remain unchanged with the weaker assumption that $\Delta X^i_t, \Delta Y_t^i$, which is closer to the assumption the authors make in their paper. So I changed it.
    \item \textbf{Formalization of mathematical derivation}. The helpful referee report stresses the important of the mathematical proof in the commentary for interpreting the original paper. Therefore, I combined the derivation into a formal statement of a proposition and its proof, highlighting the importance of the result.
\end{itemize}


\end{document}
